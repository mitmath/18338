\documentclass{article}
\usepackage{amsmath}
\usepackage{amsfonts}
\usepackage{color}
\usepackage[utf8]{inputenc}
\usepackage[a4paper,left=2.5cm,right=2.5cm,top=2cm,bottom=2cm]{geometry}

\title{Homework 2, Fall 2025 18338}
\author{Due 9/16 11:59pm}
\date{}
\begin{document}
\maketitle

\noindent{\large\color{blue} Please submit your homework via canvas.mit.edu. \\
If you are submitting .jl or .ipynb files, you must additionally submit .html or .pdf file that captures running notebook or code.}

\subsection*{Reading and Notes}

Read chapters 7, 13, 14, 21 of the class notes
(The lecture notes  can be found in Piazza: http://piazza.com/mit/fall2025/18338.). 
Optional, but if you do this I'll get to know you : Provide  feedback especially high level style and where things did not make sense, in addition to spelling or technical errors.


\subsection*{Problem sets}
\begin{enumerate}
	\item Do at least four of the following problems. 
\begin{itemize}
\item Exercise 7.1 (page 124)
\item Exercise 7.2 (page 124)
\item Exercise 7.3 (page 126)
\item Exercise 7.4 (page 129)
\item Exercise 7.5 (page 130)
\item Exercise 7.6 (page 131)
\end{itemize}
\item See Lanczos tridiagonalization at page 254. Prove that the following sequence readily solves for the red variables in the Lanczos tridiagonalization algorithm:
\begin{equation*}
\hat{w}_k = Dq_k, \alpha_k = \hat{w}_k^T q_k, w_k = \hat{w}_k - \alpha_k q_k - \beta_{k-1}q_{k-1}, \beta_k = ||w_k||, q_{k+1} = w_k/\beta_k
\end{equation*}
\item Do at least four of the following problems. 

\begin{itemize}
\item Exercise 13.1 (page 263)
\item Exercise 13.2 (page 264)
\item Exercise 13.3 (page 264)
\item Exercise 13.4 (page 265)
\item Exercise 13.5 (page 265)
\item Exercise 13.6 (page 265)
\end{itemize}



\end{enumerate}


\end{document}


